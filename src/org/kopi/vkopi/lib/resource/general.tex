%%
%% Copyright (c) 1990-2009 kopiRight Managed Solutions GmbH
%%
%% This program is free software; you can redistribute it and/or modify
%% it under the terms of the GNU General Public License version 2
%% as published by the Free Software Foundation.
%%
%% This program is distributed in the hope that it will be useful,
%% but WITHOUT ANY WARRANTY; without even the implied warranty of
%% MERCHANTABILITY or FITNESS FOR A PARTICULAR PURPOSE. See the
%% GNU General Public License for more details.
%%
%% You should have received a copy of the GNU General Public License
%% along with this program; if not, write to the Free Software
%% Foundation, Inc., 59 Temple Place, Suite 330, Boston, MA 02111-1307 USA
%%
%% $Id$
%%

\chapter{Einleitung}

Das vorliegende Dokument dient als Einf\"{u}hrung und zur Bedienung bzw.
Handhabung von mit KOPI entwickelten Programmen.\\

Dieses Dokument ist in mehrere Kapitel unterteilt.\\

Das erste Kapitel, die Einleitung, dient zur allgemeinen Erkl\"{a}rung und soll
dem Leser=Benutzer helfen, sich einen \"{U}berblick \"{u}ber das ihm
vorliegende Dokument zu verschaffen. \\

Im zweiten Kapitel soll dem Leser ein \"{U}berblick
\"{u}ber die ihm zur Verf\"{u}gung stehende Hardware vermittelt werden. Weiters
erh\"{a}lt der Leser eine kurze Erkl\"{a}rung \"{u}ber dessen korrekte Nutzung.\\

Im dritten Kapitel werden  allgemeine Begriffe der Software, die Handhabung der
Programm-Auswahl sowie die Tastaturbelegung erkl\"{a}rt, die f\"{u}r mit KOPI entwickelte Programme gelten. Diese Begriffe sind die Basisvoraussetzung f\"{u}r die
Bedienung des Systems (Programms).\\

Die restlichen Kapitel (vier bis achzehn) erl\"{a}utern schlie{\ss}lich den
Programmaufbau, die Masken und Felder des Programmes und die Bedienung der
einzelnen Module, die dem Benutzer in diesem Programm zur Verf\"{u}gung stehen.\\

\cleardoublepage

\chapter{Hardware allgemein}

\cleardoublepage

\section{Bildschirm}\index{Bildschirm}

In einer Computeranwendung (Programm) erf\"{u}llt der Bildschirm
grunds\"{a}tzlich immer die Funktion eines Anzeigemediums. Der Benutzer
kann sich auf einem Bildschirm verschiedene Masken, Grafiken oder
andere Darstellungen anzeigen lassen.\\

Da das Anzeigen von Masken, Grafiken und Darstellungen meistens statisch ist, d.h. nicht wie bei einem Fernseher in bewegten Bildern stattfindet,
besteht bei einem Bildschirm die Gefahr, da{\ss} sich das angezeigte Bild
in die Bildr\"{o}hre brennt und das Ger\"{a}t dadurch fr\"{u}hzeitig
abgenutzt bzw. kaputt wird. \\

Um dieser eventuellen Gefahr, die durch die Bedienung eines oder
mehrerer Benutzer entsteht, Rechnung zu tragen, besitzen die meisten
am Markt erh\"{a}ltlichen Bildschirme eine sogenannte Sleeper-Funktion.
Diese Funktion aktiviert den Bildschirmschoner oder verdunkelt, nach
einer vom Benutzer festgelegten Anzahl von Minuten, automatisch den
Bildschirm. Durch eine Bewegung mit der Maus oder durch einen einfachen Druck
auf eine Taste der Tastatur wird die Sleeper-Funktion wieder au{\ss}er Kraft gesetzt.\\
Achtung! Das Programm interpretiert diesen Tastendruck als neue Eingabe.
Daher:

\vspace{0.5cm}
\begin{bf}
\begin{it}
Ein dunkler Bildschirm mu{\ss} kein abgeschalteter Bildschirm sein!
(Kontrollampe des Bildschirms beachten). Zur \"{U}berpr\"{u}fung,
ob bei dem Bildschirm die Sleeper-Funktion aktiv ist oder ob der
Bildschirm abgeschaltet ist, die Maus bewegen oder eine Taste dr\"{u}cken.\\
Bitte verwenden Sie dazu aber nur unbedenkliche Tasten wie z.B.
die \Taste{Hochstelltaste}, auf keinen Fall d\"{u}rfen Sie
zu diesem Zweck die \Taste{Enter}-, \Taste{Esc}- oder auch eine
Funktionstaste verwenden, da diese eventuell Aktionen ausl\"{o}sen k\"{o}nnten.
\end{it}
\end{bf}
\vspace{0.5cm}

Besitzt Ihr Bildschirm keine sogenannte Sleeper-Funktion
(Verdunkelungsmechanismus), so achten Sie bitte darauf, da{\ss} der
Schirm nicht \"{u}ber l\"{a}ngere Zeit eingeschalten bleibt, ohne
da{\ss} an ihm gearbeitet wird.

\subsection{Der Cursor(Eingabepunkt)}

Der Cursor (Eingabepunkt) zeigt die Stelle am Bildschirm bzw. in einer
Maske an, an der sich der Benutzer gerade befindet. Der Cursor wird
meistens durch einen kleinen blinkenden Strich oder Block dargestellt.

\section{Tastatur}\index{Tastatur}

Die Funktionsweise einer Tastatur ist der einer Schreibmaschine sehr
\"{a}hnlich. Die Abweichungen werden im weiteren Verlauf dieses
Punktes kurz beschrieben.\\

Die Tastatur ist in vier unterschiedliche Bl\"{o}cke unterteilt, und zwar in

\begin{itemize}
\item den Schreibmaschinen - Tastaturblock (links),
\item den Kontrollblock (mitte),
\item den Zahlenblock (rechts) und
\item den Funktionstastenblock (oben).
\end{itemize}

Die Zeichen, die sich \"{u}ber manchen Tasten befinden, z.B. die eckigen
Klammern auf der Taste \Taste{8} und \Taste{9} sind nur mittels der Taste
\Taste{Shift} (Hochstelltaste) erreichbar.\\
Weitere wichtige Tasten werden in den nachfolgenden Punkten kurz beschrieben.

%%\raisebox{-280truept}[47truept][0truept]{\makebox[-120truept][r]{\special{psfile=tast.ps}}}

\subsection{Die \protect\Taste{Enter} oder \protect\Taste{Return} Taste}

Die wichtigste Taste in den meisten Programmen ist gro{\ss}e Taste
mit dem abgewinkelten Pfeil. Diese Taste wird als \Taste{Enter} oder
\Taste{Return} Taste bezeichnet. Mit dieser Taste kann ein Benutzer

\begin{itemize}
\item eine oder mehrere Eingaben best\"{a}tigen,
\item eine Zeile beenden bzw. in die n\"{a}chste Zeile springen,
\item einen oder mehrere Eintr\"{a}ge aus einer Liste ausw\"{a}hlen
\end{itemize}

und vieles mehr.\\

Die \Taste{Enter} Taste befindet sich in der rechten H\"{a}lfte des
normalen Schreibmaschinen- bzw. Tastaturblockes. Auf einer Tastatur
mit einem Zahlenblock befindet sich meistens in diesem Zahlenblock
eine zweite \Taste{Enter} Taste.

\subsection{Die \protect\Taste{Esc} Taste}
Die \Taste{Esc} (``Escape'' = ``entkommen'') Taste ist im Prinzip das
Gegenst\"{u}ck zur \Taste{Enter} Taste. Mit dieser Taste kann der
Benutzer

\begin{itemize}
\item die meisten Dinge widerrufen,
\item aus Programmen oder Programmteilen aussteigen oder
\item Eingaben r\"{u}ckg\"{a}ngig machen.
\end{itemize}

Die \Taste{Esc}-Taste befindet sich am linken oberen Ende der Tastatur neben
dem Funktionstastenblock.

\subsection{Gro{\ss}buchstaben- bzw. Hochstelltaste \protect\Taste{Shift}}

Durch Dr\"{u}cken der Hochstelltaste (auch \Taste{Shift}-Taste genannt),
wird es dem Benutzer erm\"{o}glicht, Gro{\ss}buchstaben bzw. Sonder-
und Satzzeichen einzugeben. Die Hochstelltaste funktioniert genauso
wie auf einer Schreibmaschine. Die \Taste{Shift} Taste ist zweimal
(links unten und rechts unten im normalen Block) angebracht.\\

Die Fixierung der Hochstelltaste ist mit der Taste \Taste{Caps}, welche
sich \"{u}ber der linken \Taste{Shift} Taste befindet, m\"{o}glich.\\

Ob die Hochstelltaste fixiert ist, erkennt man am dazugeh\"{o}rigen
Kontroll\"{a}mpchen - leuchtet es, so ist die Hochstelltaste fixiert.
Um sie wieder zu entriegeln, bet\"{a}tigt man sie einfach ein zweites Mal.\\

(HINWEIS: Ist die Verriegelung eingeschaltet, sind trotzdem noch die Zahlen und
nicht die jeweiligen Sonderzeichen aktiviert.)


\subsection{L\"{o}schtasten}

Es gibt zwei M\"{o}glichkeiten, einzelne Zeichen zu l\"{o}schen.

Erstens die sogenannte R\"{u}ckschrittaste (auch \Taste{Backspace} Taste
genannt), die auch auf den meisten Schreibmaschinen vorhanden ist.
Diese Taste befindet sich ganz rechts oben am normalen Tastaturblock
und l\"{o}scht das Zeichen vor, d.h. links vom Cursor.

Die zweite M\"{o}glichkeit, ein Zeichen zu l\"{o}schen, ist die Taste
\Taste{Entf} (``Entf'' steht f\"{u}r ``entfernen''), diese befindet sich
im Kontrollblock. Sie l\"{o}scht das Zeichen, das sich genau an der
Cursor-Position befindet. Bei L\"{o}schung eines Zeichens r\"{u}ckt
der nachfolgende Text automatisch um ein Zeichen nach.

\subsection{Cursortasten, Positionstasten} \index{Cursortasten}

Zum Bewegen des Cursors im Bildschirm oder in einer Bildschirmmaske,
ohne eine Eingabe zu t\"{a}tigen, kann man grunds\"{a}tzlich die
Cursortasten benutzen. Diese Tasten sind wie ein Kreuz angeordnet
und befinden sich im Kontrollblock. Sie bewegen den Cursor um jeweils
ein Zeichen nach links, rechts, unten oder oben.\\
Sollte der Cursor in der gew\"{u}nschten Richtung nicht mehr
weiterbewegt werden k\"{o}nnen, ert\"{o}nt in den meisten F\"{a}llen
ein akustisches Warnsignal.\\
Es gibt neben den normalen Cursortasten auch noch spezielle
Positionstasten, die aber nicht in jedem Programm jederzeit
verf\"{u}gbar sind und manchmal auch ein wenig ver\"{a}nderte
Funktionen besitzen.

\begin{enumerate}
\item Tabulator \index{Tabulator} \\

Die Tabulatortaste, diese befindet sich wie bei der Schreibmaschine
\"{u}ber der Hochstellfixierungstaste \Taste{Caps}, bewegt den Cursor um eine
Tabulatorposition weiter nach rechts, bzw. bei gleichzeitigem Dr\"{u}cken der
Taste \Taste{Shift} um eine Position nach links.\\
In einer Bildschirmmaske kommt man mit der Tabulatortaste um ein Feld
vorw\"{a}rts bzw. um ein Feld r\"{u}ckw\"{a}rts.

\item Positionstasten \index{Positionstaste} \\

Im Kontrollblock befinden sich die restlichen Positionstasten.
Mit der Taste \Taste{Pos1} kann man zum Anfang bzw. mit der Taste
\Taste{Ende} zum Ende eines bearbeiteten Bereiches springen.
Die Tasten \Taste{Bild $\uparrow$} und \Taste{Bild $\downarrow$}
agieren als Pendant zu den jeweiligen Cursortasten, nur, da{\ss} sie
nicht nur eine Zeile, sondern um eine ganze Bildschirmseite
weiterspringen. Auch bei diesen Tasten gibt es abh\"{a}ngig von den
verschiedenen Programmen unterschiedliche Belegungen.

\item Einf\"{u}ge-Taste \index{Einf\"{u}ge-Taste} \\

Um zwischen dem Einf\"{u}ge- und \"{U}berschreibmodus umzuschalten,
wird die Taste \Taste{Einfg} benutzt. Mit dieser Taste kann der
Benutzer zwischen den M\"{o}glichkeiten w\"{a}hlen, ob der
nachfolgende Text entweder \"{u}ber\-schrieben oder verschoben
wird, d.h., da{\ss} ein neu eingegebener Text eingef\"{u}gt bzw.
\"{u}berschrieben werden kann. In welchem Modus (Einf\"{u}ge- oder
\"{U}berschreibmodus) sich der Benutzer gerade befindet, wird
meistens durch eine kurze Erkl\"{a}rung oder Abk\"{u}rzung in der
untersten Bildschirmzeile angezeigt.
\end{enumerate}

\subsection{Zahlenblock} \index{Zahlenblock}

Mit dem Zahlen- oder Zehnerblock, welcher sich im rechten Viertel der
Tastatur befindet, kann der Benutzer schneller die Eingabe von
Zahlen, einem Datum oder \"{a}hnlichem bewerkstelligen. Aktiviert
wird der Block mit der im linken oberen Eck befindlichen Taste
\Taste{Num}. Ob der Zahlenblock aktiviert ist oder nicht, kann
der Benutzer durch ein Kontroll\"{a}mpchen, welches sich oberhalb der
Taste befindet, sehen (leuchtet es - ist der Block aktiviert). Sollte
der Block nicht aktiviert sein (das Kontroll\"{a}mpchen leuchtet nicht), verhalten sich die Tasten im Zahlenblock in \"{a}hnlicher Weise
wie die des Kontrollblocks.\\
Im Zahlenblock befindet sich auch eine Taste \Taste{Enter}, die
dem gleichen Zweck dient, wie die normale gro{\ss}e Taste \Taste{Enter},
nur in ganz wenigen Ausnahmef\"{a}llen kann diese Taste auch eine
eigene Funktion erf\"{u}llen.

\subsection{Funktionstasten} \index{Funktionstasten}

Der Funktionstastenblock, welcher aus 12 Funktionstasten besteht,
befindet sich am oberen Rand der Tastatur, diese Funktionstasten
sind in jedem Programm anders belegt. Die einzige Funktionstaste,
die in jedem Programm gleich belegt ist, ist die Taste \Taste{F1}.
Diese Taste hat in jedem Programm eine {\bf Hilfefunktion}.

\section{Maus}\index{Maus}

Eine andere M\"{o}glichkeit, sich am Bildschirm zu bewegen, bietet die
Maus. Anstelle des Benutzens von verschiedenen Tasten kann man mit der Maus
durch Anklicken die Position des Cursers ver\"{a}ndern, mit gehaltener
Maustaste einzelne W\"{o}rter bzw. Textstellen markieren oder in der
Men\"{u}zeile verschiedene Aktionen durch Mausklick ausw\"{a}hlen (=Funktionstasten der Tastatur).

\section{Zentraleinheit (Computer)}\index{Zentraleinheit}

Die Zentraleinheit kann entweder ein Server (PC oder Gro{\ss}rechner), an
dem ein oder mehrere Arbeitspl\"{a}tze angeschlossen sind (Mehrplatzsystem),
oder ein einzelner alleinstehender Computer (Einplatzsystem) sein.\\
Die Zentraleinheit bildet das Nervenzentrum des Systems, welches
die Daten an die verschiedenen Medien (Bildschirm, Drucker, usw.)
verteilt. Wenn es sich bei der Zentraleinheit um einen Server handelt, an dem
mehrere Benutzer angeschlossen sind, ist es somit von extremer Wichtigkeit, da{\ss} dieser niemals ausgeschaltet wird, bevor
man sich nicht davon \"{u}berzeugt hat, da{\ss} keiner der Benutzer noch
arbeitet. Bei einem einzelnen Computer kann man die Zentraleinheit
dann abschalten, wenn  der momentan darauf arbeitende Benutzer
sicher ist, da{\ss} alle vorhandenen Daten oder Arbeiten ausgef\"{u}hrt
worden sind.

\vspace{0.5cm}

\begin{bf}
\begin{it}
ACHTUNG: Nie den Ein-Aus-Schalter der Zentraleinheit bet\"{a}tigen, wenn
das System l\"{a}uft. Zuerst mu{\ss} das Abschalt-Programm gestartet werden.
Erst danach darf der Computer abgeschaltet werden. Das Abschalten der
Zentrale sollte nur von einer Person (System-Administrator) durchgef\"{u}hrt
werden.
\end{it}
\end{bf}

\section{Drucker} \index{Drucker}

Der Drucker ist das Ausgabemedium, mit dessen Hilfe der Benutzer eine
oder mehrere Listen, Grafiken oder Darstellungen ausdrucken kann. Ein
Drucker empf\"{a}ngt die Daten, die er ausgeben (ausdrucken) soll, von
verschiedenen Programmen, welche auf einem Server (Computer) oder PC
(Personal Computer) installiert sind, und kann von einem oder mehreren
Benutzern bedient werden.\\
Der Benutzer kann Daten aber nur dann am Drucker ausgeben, wenn dieser
auf Empfang (d.h. {\bf ONLINE}) ist. Das Gegenteil dazu ist,
da{\ss} der Drucker nicht auf Empfang (d.h. {\bf OFFLINE}) ist. Diese zwei
Betriebszust\"{a}nde werden entweder durch ein Display oder durch eine
Taste am Drucker angezeigt.\\
Hat ein Benutzer einen Druckauftrag (Liste, Auswertung, Darstellung)
gestartet und der Drucker gibt keine Daten aus, so sollte der
Benutzer zuerst den Betriebszustand des Druckers (ONLINE - auf
Empfang, OFFLINE - nicht auf Empang) kontrollieren.\\
Es hat keinen Sinn, einen Druckauftrag, welcher von einem Benutzer
gestartet, aber vom Drucker, der nicht auf Empfang war, noch nicht ausgegeben
wurde, nochmals zu starten, da der Druckauftrag zweimal ausgegeben wird, wenn
der Drucker dann wieder auf Empfang ist.

\section{Laufwerk} \index{Laufwerk} \index{Diskettenlaufwerk} \index{Kassettenlaufwerk}

Es gibt verschiedene Arten von Laufwerken:

\begin{itemize}
\item Diskettenlaufwerk
\item Kassettenlaufwerk
\begin{itemize}
\item Streamer Tape
\item DAT Tape
\end{itemize}
\item Wechselplattenlaufwerk
\item Optisches Laufwerk
\end{itemize}

Die Hauptaufgabe aller dieser Arten von Laufwerken ist die Datensicherung.
Eine regelm\"{a}{\ss}ige, m\"{o}glichst t\"{a}gliche Datensicherung ist
dringend anzuraten, um f\"{u}r den Fall ger\"{u}stet zu sein, da{\ss} der
Server(Computer) oder der PC (Personal Computer) bzw. die Festplatte einen
nicht r\"{u}ckg\"{a}ngig zu machenden Fehler aufweist. Denn nur mit einer
ordnungsgem\"{a}{\ss}en Datensicherung kann der alte Programm- und Datenbestand
wieder hergestellt werden.

\cleardoublepage

\chapter{Basisbegriffe-Software}

Die folgenden Kapitel/Unterkapitel beschreiben die Software: die
Programm-Auswahl, den Aufbau und die Handhabung der Masken, das
Ordner-System, nach dem die Programm-Auswahl aufgebaut ist, sowie die
Handhabung der Eingabemasken. Au{\ss}erdem wird auf die verschiedenen
Bedeutungen der Funktionstasten eingegangen, wobei die Tasten
eingerahmt dargestellt werden (z.B. kennzeichnet \Taste{F1} die
Funktionstaste F1).  Eine Tastenkombination (d.h. zwei Tasten
m\"{u}ssen gleichzeitig gedr\"{u}ckt werden) wird durch die zwei
Tasten und ein ``+'' dargestellt (z.B. \Taste{shift}+\Taste{TAB}
bedeutet, da{\ss} die Taste \Taste{shift} gleichzeitig mit der Taste
\Taste{TAB} gedr\"{u}ckt werden mu{\ss}).\\ Auf die Erl\"{a}uterung,
wie sich jeder Benutzer seine ``Favoriten'' anlegen kann, folgen
abschlie{\ss}end Informationen \"{u}ber die verschiedenen Typen von
Feldern und m\"{o}gliche Meldungen von Seiten des Systems.

\cleardoublepage

\section{Programm-Auswahl} \index{Men\"{u}s}

Ruft man das System auf, dann \"{o}ffnet
sich eine Maske bzw. Fenster mit der sogenannten Programm-Auswahl.
Diese bietet dem Benutzer eine Auswahlm\"{o}glichkeit an, in deren Rahmen er
die gew\"{u}nschte Auswahl treffen kann. \\
\\
Der {\bf Aufbau} der Programm-Auswahl gestaltet sich generell in Form eines
Ordnersystems. Man kann demnach die Programm-Auswahl unterteilen in 

\begin{itemize}
\item Ordner
\item Unterordner und
\item Eingabemasken.
\end{itemize}

In  den nun fogenden Unterkapiteln  wird speziell auf diese Punkte
eingegangen, nachdem zuerst einige allgemeine Erl\"{a}uterungen zu den Masken
und der Programm-Auswahl angef\"{u}hrt werden.

\subsection{Allgemeine Erl\"{a}uterungen}

Ganz oben auf der Maske der Programm-Auswahl befinden sich drei
Maskenmen\"{u}s:
``Datei'', ``Bearbeiten'' und ``Hilfe''.\\
\\
Darunter befindet sich eine Funktionsleiste, die, abh\"{a}ngig davon, in
welchem Programm man sich befindet, aus verschieden vielen ``Druckkn\"{o}pfen'' mit
einer Kurzbenennung besteht. Diese Druckkn\"{o}pfe stellen verschiedene
Funktionen oder Aktionen dar. Wenn man sich mit der Maus auf einen
``Druckknopf'' stellt und die Maus dann nicht mehr bewegt, erscheint eine
n\"{a}here Beschreibung der Funktion dieses Druckknopfes. Ein paar Beispiele
daf\"{u}r: ``Beenden'' (Beenden des Programmes), ``Aufrufen'' (ein Programm
aufrufen), ``\"{O}ffnen'' und ``Schlie{\ss}en'' (ein bestimmtes Verzeichnis
\"{o}ffnen oder schlie{\ss}en). Grau dargestellte Kn\"{o}pfe k\"{o}nnen nicht
bet\"{a}tigt werden.\\
\\
Am unteren Rand des Fensters befindet sich eine Leiste mit Hilfstexten, die
zus\"{a}tzliche Bearbeitungshinweise enth\"{a}lt: steht man zum
Beispiel auf dem Untermen\"{u} ``Stammdaten Verwaltung'', so scheint in dieser
Leiste als Zusatzinformation auf: ``Hier k\"{o}nnen die verschiedenen
Stammdaten verwaltet werden.''\\
\\
Ganz allgemein ist hier zu sagen, da{\ss} der Benutzer 3 M\"{o}glichkeiten hat,
in einer Programm-Auswahl Aktionen zu setzen.

\begin{itemize}
\item Der Benutzer verwendet die Funktionstasten auf der Tastatur, auf die
weiter unten n\"{a}her eingegangen wird (z.B.: F7=Speichern).
\item Der Benutzer arbeitet mit der Maus und klickt auf die sich in der
Funktionsleiste befindlichen Druckkn\"{o}pfe.
\item Die dritte M\"{o}glichkeit besteht darin, mit der Maus auf die ganz oben
angef\"{u}hrten Maskenmen\"{u}s zu gehen und unter ``Datei'', ``Bearbeiten''
oder ``Aktionen'' die jeweils gew\"{u}nschte Aktion auszuw\"{a}hlen und zu
setzen. Hinzuzuf\"{u}gen ist noch, da{\ss} hier auch die zu der
jeweiligen Aktion alternativ zu verwendenden Funktionstasten angezeigt werden. 
\end{itemize}

In der Programm-Auswahl kann der Benutzer zwischen den
Anwendungspunkten  mit Hilfe der Maus oder mit den Tasten
\Taste{$\downarrow$} und \Taste{$\uparrow$} hin- und herspringen, wobei die
momentan g\"{u}ltige Auswahl farbig unterlegt wird.\\

Die Programm-Auswahl kann der Benutzer entweder unter dem Maskenmen\"{u}
``Datei''-Beenden  oder durch
Anklicken des Druckknopfes ``Beenden'' in der Funktionsleiste
verlassen.

\vspace{0.5cm}

\begin{Keys}{Zusammenfassung der wichtigsten Tasten}
\StdKey{$\uparrow$}{}{Aktiviert den vorhergehenden Auswahlpunkt}
\StdKey{$\downarrow$}{}{Aktiviert den n\"{a}chsten Auswahlpunkt}
\StdKey{F1}{Hilfe}{Blendet ein Hilfefenster zum aktiven Auswahlpunkt ein.}\\
\StdKey{Enter}{}{Ausw\"{a}hlen des aktiven Auswahlpunktes.}
\end{Keys}

\subsection{Ordner und Unterordner}

Um einen Ordner zu \"{o}ffnen, klickt man entweder auf das kleine
voranstehende + (plus) K\"{a}stchen  oder bet\"{a}tigt die  \Taste{Enter}
Taste, vorausgesetzt, man hat diesen Punkt ausgew\"{a}lt. Eine dritte M\"{o}glichkeit
w\"{a}re, wie bereits erw\"{a}hnt, auf den Druckknopf ``\"{O}ffnen'' in der
Funktionsleiste zu klicken. Der ``Ordner'' \"{o}ffnet sich und man bekommt
entweder verschiedene Eingabemasken zur Auswahl oder nochmals Unterordner, die
sich wiederum auf die gleiche Art und Weise \"{o}ffnen lassen. Da{\ss} ein
Ordner ge\"{o}ffnet ist, erkennt man daran, da{\ss} aus dem + K\"{a}stchen ein
- (minus) K\"{a}stchen geworden ist. Um einen Ordner wieder zu schlie{\ss}en,
klickt man auf das - (minus) K\"{a}stchen oder bet\"{a}tigt nochmals die Taste
\Taste{Enter}.\\

\subsection{Handhabung von Eingabemasken}

\"{A}hnlich wie bei der Programm-Auswahl befinden sich auch in den Eingabemasken ganz
oben am Fenster die Maskenmen\"{u}s, in diesem Falle aber vier:
``Datei'', ``Bearbeiten'' und ``Aktionen'' und``Hilfe''.\\
\\
Darunter befindet sich wieder eine Funktionsleiste, die, abh\"{a}ngig davon, in
welchem Programm man sich befindet, aus verschieden vielen ``Druckkn\"{o}pfen''
mit einer Kurzbenennung besteht. Diese Druckkn\"{o}pfe stellen verschiedene
Funktionen oder Aktionen dar. Wenn man sich mit der Maus auf einen Druckknopf stellt und die Maus dann nicht mehr bewegt, erscheint eine
n\"{a}here Beschreibung der Funktion dieses Druckknopfes.\\
\\
Am unteren Rand des Fensters der Eingabemaske befindet sich eine Leiste,
die zus\"{a}tzliche Bearbeitungshinweise zu den einzelnen Feldern
enth\"{a}lt.\\
\\
Um eine Eingabemaske zu \"{o}ffnen, stellt man sich auf die ausgew\"{a}hlte
Maske und dr\"{u}ckt die Taste \Taste{Enter} oder macht darauf einen
Doppelklick mit der Maus oder bet\"{a}tigt den Druckknopf `` Aufrufen''.   \\
\\
Eine Eingabemaske besteht aus einer bestimmten Menge von Eingabefeldern.\\
\\
So wie bei der Programm-Auswahl ist in der Eingabemaske immer nur ein Feld
aktiv. Dies ist das Feld, in dem sich der Cursor momentan befindet. Prinzipiell
unterscheidet man zwischen zwei Arten von Feldern:

\begin{itemize}
\item Felder, die vom Benutzer ver\"{a}ndert werden k\"{o}nnen,
\item Felder, die vom Benutzer nicht ver\"{a}ndert werden k\"{o}nnen,
\end{itemize}
wobei hinzuzuf\"{u}gen ist, da{\ss} diese entweder einzeilig oder zweizeilig sein k\"{o}nnen.\\
\\
In einem Eingabefeld kann sich der Benutzer mit Hilfe der Maus oder den 4
Kursortasen (Pfeiltasten) bewegen.

\begin{itemize}
\item \Taste{$\rightarrow$} eine Stelle(Buchstabe) nach rechts
\item \Taste{$\leftarrow$} eine Stelle(Buchstabe) nach links
\item bei mehrzeiligen Feldern: \Taste{$\downarrow$} n\"{a}chste Zeile
\item bei mehrzeiligen Feldern: \Taste{$\uparrow$} vorhergehende Zeile
\end{itemize}

In der Eingabemaske kann sich der Benutzer mit Hilfe der Maus oder der
folgenden Tasten zwischen den von ihm zu ver\"{a}ndernden Eingabefeldern
bewegen.

\begin{itemize}
\item \Taste{TAB} zum n\"{a}chsten Feld
\item \Taste{shift}+\Taste{TAB} zum vorherigen Feld
\item \Taste{Enter} Taste bewegt den Kursor ins n\"{a}chste Feld \"{a}hnlich
	wie \Taste{TAB}. Bei mehrzeiligen Feldern jedoch bewirkt diese
	Taste einen Zeilenumbruch
\end{itemize}

Befinden sich in einer Eingabemaske mehrere Felder gleicher Art
untereinander (mehrzeilige Eintr\"{a}ge), so werden die Tasten
\Taste{Bild$\uparrow$} und \Taste{Bild $\downarrow$} dazu verwendet, um
von einem Eintrag in den n\"{a}chsten bzw. in den vorhergegangenen
Eintrag zu gelangen.\\
\\


\begin{Keys}{Die Bedeutung(-en) der Funktionstasten}\\
Alle Aktionen, die aus einer Eingabemaske heraus von einem
Benutzer ausgef\"{u}hrt werden k\"{o}nnen, sind durch eine
Funktionstaste bezeichnet. Bei den folgenden Beschreibungen handelt es sich um Standardbelegungen, deren
Bedeutungen sich aber abh\"{a}ngig vom Programm ver\"{a}ndern k\"{o}nnen:


\StdKey{F1}{}{Hilfe}\\
{Durch das Bet\"{a}tigen der Taste \Taste{F1} erh\"{a}lt der Benutzer ein
Hilfefenster eingeblendet. Dieses zeigt zu dem jeweiligen Feld der
Eingabemaske, in der sich der Benutzer momentan befindet, einen
Hilfetext sowie auch die Beschreibung der m\"{o}glichen Funktionen an.\\
Jedes Hilfefenster kann durch die Taste \Taste{Esc} verlassen werden.}

\StdKey{F2}{}{Die Bedeutung von  \Taste{F2} ist feldbezogen, das hei{\ss}t, sie wechselt
von Feld zu Feld. Gibt es bei einem Feld Auswahlm\"{o}glichkeiten, auch
erkennbar daran, da{\ss} sich am Ende eines Eingabefeldes ein kleiner
Druckknopf mit einer Lupe befindet, dann erlaubt dem Benutzer das Dr\"{u}cken
der Funktionstaste  \Taste{F2} bzw. das Anklicken der Lupe, da{\ss} das Ergebnis in Form
einer Liste angezeigt wird. In diesen Listen kann sich der Benutzer entweder
mit den Tasten \Taste{$\uparrow$} und \Taste{$\downarrow$} von einer Zeile zur
anderen bewegen oder er kann mit den Tasten \Taste{Bild$\uparrow$} und
\Taste{Bild $\downarrow$} von einer Seite zur anderen springen.\\
Wie im Anwendungsmen\"{u} erfolgt die Auswahl eines Punktes der Liste durch
die Taste \Taste{Enter}, d.h. der Wert des aktiven Punktes wird in
das Feld der Eingabemaske \"{u}bernommen, in dem sich der Benutzer momentan
befindet.\\
Die Taste \Taste{Esc} blendet die Liste aus, ohne eine Auswahl zu
treffen. \\
Bei leeren Uhrzeit- und Datumsfeldern setzt die Taste \Taste{F2} die aktuelle
Uhrzeit bzw. das aktuelle Datum ein. Befindet sich bereits ein Datum in dem
Feld, bewirkt ein nochmaliges Dr\"{u}cken der Taste \Taste{F2} das Erscheinen
eines Kalenders, mithilfe dessen man dann das momentan eingetragene Datum
\"{a}ndern kann.\\
Au{\ss}erdem hat man die M\"{o}glichkeit, wenn schon ein Datum im Feld steht,
mit den Tasten  \Taste{$\uparrow$} und \Taste{$\downarrow$} zeitm\"{a}{\ss}ig
nach vorne oder nach hinten zu bl\"{a}ttern.}
\StdKey{F3}{}{Abbrechen}\\
{Die Eingaben in die Maske werden gel\"{o}scht und  die Eingabemaske wird
wieder in den urspr\"{u}nglichen Modus bei Aufruf (Suchmodus) versetzt.}
\StdKey{F4}{}{Anlegen}\\
{Die Maske wird in den Anlege-Modus versetzt und es wird hiermit
erm\"{o}glicht, neue Datens\"{a}tze anzulegen.}
\StdKey{F5}{}{Befindet sich der Benutzer im Anlege-Modus, dann bedeutet F5
L\"{o}schen, befindet er sich hingegen im Such-Modus, werden mit Hilfe der
Funktionstaste F5 die Suchoperatoren aufgerufen. Die Suchoperatoren
erm\"{o}glichen komplexere Abfragen (z.B. alle Artikel, deren Preis
gr\"{o}{\ss}er als 100 ist).}
\StdKey{F6}{}{Suche starten und Bl\"{a}ttern}\\
{Der Benutzer hat die M\"{o}glichkeit, einen bestimmten Eintrag zu suchen,
indem er alle gespeicherten Eintr\"{a}ge vom ersten bis zum letzten Datensatz
mit den Tasten  \Taste{Bild$\uparrow$} und \Taste{Bild $\downarrow$} durchbl\"{a}ttert.}
\StdKey{F7}{}{Speichern}\\
{Der bearbeitete Datensatz wird abgespeichert.}
\StdKey{F8}{}{Diese Taste hat zwei Bedeutungen: \\
{\bf Erstens} kann sich der Benutzer damit eine Liste aufrufen lassen. Befindet er sich in einem
bestimmten Programm (z.B.: Firmen - Stammdaten), erm\"{o}glicht es ihm die
Taste  \Taste{F8}, sich eine Liste mit allen gespeicherten Eintr\"{a}gen
auflisten zu lassen. Dabei ist es m\"{o}glich, sich s\"{a}mtliche Eintr\"{a}ge
anzeigen zu lassen (alle Felder bleiben frei) oder durch Ausf\"{u}llen gewisser Felder die Auswahl enger
zu gestalten. Mit den Tasten  \Taste{$\uparrow$} und
\Taste{$\downarrow$} kann man sich in dieser Liste bewegen und durch die Taste
\Taste{Enter} kann der ausgew\"{a}hlte Eintrag direkt in die Eingabemaske
\"{u}bernommen werden. Nat\"{u}rlich kann man daf\"{u}r auch die Maus benutzen,
indem man einfach den gew\"{u}nschten Eintrag anklickt.\\
 Verlassen ohne eine Auswahl getroffen zu haben wird diese Liste mit der Taste \Taste{Esc}.\\
{\bf Zweitens} bewirkt die Taste \Taste{F8} bei Eingabemasken, die in
verschiedene Bl\"{o}cke eingeteilt sind, einen Wechsel von einem Block in den
anderen (z.B. gibt es bei Auftr\"{a}gen den Auftragskopf und die
Auftragspositionen, und  die Taste \Taste{F8} erm\"{o}glicht es, schnell von einem Block in den anderen zu wechseln).}
\StdKey{F12}{}{Favoriten anzeigen}\\
{Die Favoriten des Benutzers werden in einem ``frei-schwebenden'' Fenster angezeigt.}
\end{Keys}

Alle Eingabemasken werden mit der Taste \Taste{Esc} oder einem Klick auf den
Druckknopf ``Beenden'' verlassen. Wurden
durch den Benutzer bereits Felder der Eingabemaske gef\"{u}llt, fragt
das Programm, ob der Ausstieg wirklich erw\"{u}nscht sei, da sonst der
Inhalt der Felder verloren geht.

\vspace{0.5cm}

\begin{Keys}{Zusammenfassung der wichtigsten Tasten}
\StdKey{TAB}{}{N\"{a}chstes Feld}
\StdKey{shift}{+ \Taste{TAB}}{Vorhergehendes Feld}
\StdKey{$\downarrow$}{}{Bei einem einzeiligen Feld: n\"{a}chstes Feld\\
	bei einem mehrzeiligen Feld: n\"{a}chste Zeile}
\StdKey{$\uparrow$}{}{Bei einem einzeiligem Feld: vorhergehendes Feld\\
	bei einem mehrzeiligen Feld: vorhergehende Zeile}
\StdKey{$\rightarrow$}{}{Kursor wird innerhalb des aktiven Feldes nach
rechts bewegt}
\StdKey{$\leftarrow$}{}{Kursor wird innerhalb des aktiven Feldes nach links bewegt}
\StdKey{Bild $\uparrow$}{}{Bei mehrzeiligen Eintr\"{a}gen vorhergehender Eintrag}
\StdKey{Bild $\downarrow$}{}{Bei mehrzeiligen Eintr\"{a}gen n\"{a}chster Eintrag}
\StdKey{Esc}{}{Ist ein Hilfefenster oder ein Auswahlmen\"{u} eingeblendet, so
bewirkt die Taste \Taste{Esc} das Ausblenden(Beenden) dieses Fensters oder
des Men\"{u}s. Sonst dient die Taste \Taste{Esc} zum Verlassen einer
Eingabemaske oder zum Beenden der meisten Verarbeitungen.}
\StdKey{Enter}{}{Bei einzeiligem Feld: n\"{a}chstes Feld; bei mehrzeiligem
Feld: Zeilenumbruch}.
\end{Keys}

\subsection{Favoriten anlegen}

Jeder Benutzer hat die M\"{o}glichkeit, f\"{u}r sich
Favoriten zu definieren und anzulegen, das hei{\ss}t, da{\ss} bestimmte
Eingabemasken/Programme, welche vom Benutzer sehr h\"{a}ufig verwendet werden, am
Bildschirm mittels eines Druckknopfes angezeit werden. Diese
M\"{o}glichkeit erlaubt es dem Benutzer, schneller zu dieser bestimmten
Anwendung zu gelangen.\\
Vorgangsweise ist folgende: das Programm, das der Benutzer gerne als Favorit
anlegen m\"{o}chte, muss im Anwendungsmen\"{u} aktiviert sein (z.B. einmal
anklicken mit der Maus), daran erkennbar, da{\ss}  der Men\"{u}eintrag blau
unterlegt ist. Anschlie{\ss}end w\"{a}hlt man im Maskenmen\"{u}  ``Bearbeiten''
den Punkt ``Favoriten anlegen'' und ein kleiner Druckknopf mit dem Namen der
gew\"{a}hlten Anwendung erscheint auf dem Bildschirm.\\
Durch Anklicken des Druckknopfes des jeweiligen Favoriten wird dann
sofort das dazugeh\"{o}rige Programm ge\"{o}ffnet.\\
Es ist noch hinzuzuf\"{u}gen, da{\ss} die Anlage von Favoriten
benutzerabh\"{a}ngig ist, was bedeutet, da{\ss} jeder Benutzer sich seine
eigenen Favoriten anlegen kann. Diese erscheinen sofort bei Starten des
Anwendungsprogrammes am Bildschirm.
\section{Felder} \index{Felder}

Das Programm unterscheidet die Felder nach der Art der Eingaben, die der Benutzer t\"{a}tigen kann.\\
Es gibt folgende Arten(Typen) von Feldern:

\begin{Fieldtypes} \index{Feldtypen} \index{Funktionstasten}
\Fieldtype{Text}{Es k\"{o}nnen alle Buchstaben bzw. Ziffern, d.h. alphanumerische Zeichen eingegeben werden}{}
\Fieldtype{Zahl}{Es k\"{o}nnen bei diesem Feldtyp nur Ziffern von '0' bis
'9' und die Vorzeichen '+' und '-' verwendet werden. Bei diesem Feldtyp wird
weiters zwischen ganzen Zahlen und Kommazahlen unterschieden. Bei Kommazahlen
kann zus\"{a}tzlich das Zeichen '.' (um Tausenderstellen zu kennzeichnen)
sowie das Zeichen ',' (komma) verwendet werden.}{}
\Fieldtype{Datum}{Die Eingabe erfolgt im Datumsformat ``TT.MM.JJJJ''. Wird
das Jahr nur zweistellig eingegeben, so wird das Datum vom System automatisch
umgewandelt. Gibt der Benutzer z.B. 1.2.10 ein, so f\"{u}llt das System auf
01.02.2010 auf.}
{\Taste{F2} (Heute): Einsetzen des aktuellen Datums, nochmaliges Dr\"{u}cken
von \Taste{F2} bewirkt die Einblendung eines Kalenders.\\
Au{\ss}erdem hat man die M\"{o}glichkeit, wenn schon ein Datum im Feld steht,
mit den Tasten  \Taste{$\uparrow$} und \Taste{$\downarrow$} zeitm\"{a}{\ss}ig
nach vorne oder nach hinten zu bl\"{a}ttern.}
\Fieldtype{Monat}{Die Eingabe erfolgt im Datumsformat ``MM.JJJJ''. Wird das
Jahr nur zweistellig eingegeben, so wird das Datum vom System umgewandelt. Gibt
der Benutzer z.B. 2.10 ein, so f\"{u}llt das System auf 02.2010 auf.}
{\Taste{F2} (Heute): Einsetzen des aktuellen Monats.}
\Fieldtype{Uhrzeit}{Die Eingabe erfolgt im Zeitformat ``MM:SS''.}
{\Taste{F2} (Jetzt): Einsetzen der aktuellen Uhrzeit.}
\Fieldtype{ja/nein}{Es kann ``ja'' oder ``nein'' eingegeben werden.
Eindeutige Abk\"{u}rzungen werden auch akzeptiert.}
{\Taste{F2} (Auswahl): Anzeigen einer Liste der erlaubten Werte}
\Fieldtype{Auswahl}{Die Eingabe wird auf eine Liste von Werten
eingeschr\"{a}nkt. Nur Werte, die in dieser Liste enthalten sind,
werden vom System akzeptiert.}
{\Taste{F2} (Auswahl): Anzeigen der Liste der erlaubten Werte.}
\end{Fieldtypes}

Hinzuzuf\"{u}gen w\"{a}re hier noch, da{\ss}, wie weiter oben bereits
erw\"{a}hnt, alternativ zur Taste \Taste{F2} auch ein Mausklick auf die Lupe
bei dem jeweiligen Feld ausreicht, um dasselbe Ergebnis zu erhalten.\\
\\
\section{Systemmeldungen} \index{Systemmeldungen}
\index{Fehlermeldungen} \index{Fragen} \index{Ablaufmeldungen}
\index{Benutzermeldungen}

Das Programm unterscheidet zwischen vier Arten von Systemmeldungen.
\begin{itemize}
\item Fehlermeldungen
\item Fragen
\item Ablaufmeldungen
\item Benutzermeldungen
\end{itemize}

{\bf Fehlermeldungen}  k\"{o}nnen wiederum in drei Gruppen
eingeteilt werden:

\begin{itemize}
\item Warnungen
\item Benutzerfehler
\item Systeminterne Fehler
\end{itemize}

Warnungen werden ausgegeben, um den Benutzer auf einen Zustand
aufmerksam zu machen, der jedoch nicht unbedingt ein Fehler sein
mu{\ss}.\\
Benutzerfehler werden dann angezeigt, wenn eine falsche Bedienung
des Programmes erfolgt ist. Die dabei angezeigte Fehlermeldung erkl\"{a}rt
den Fehler; diese Fehler k\"{o}nnen dann vom Benutzer korrigiert werden.\\
Programmfehler (Systeminterne Fehler) werden dann angezeigt, wenn
ein f\"{u}r das Programm unzul\"{a}ssiger Zustand eingetreten ist. Dabei
wird die Meldung ``Interner Fehler: bitte System--Administrator benachrichtigen'' angezeigt.\\

{\bf Fragen}  werden vom System angezeigt, wenn der Benutzer eine
Entscheidung treffen mu{\ss} (z.B. ``Ende der Bearbeitung: sind Sie
sicher? \Taste{j}/\Taste{n} ``). Diese Frage mu{\ss} mit \Taste{j} f\"{u}r
ja oder \Taste{n} f\"{u}r nein beantwortet werden, bzw. klickt man den
entsprechenden Druckknopf an.\\

{\bf Ablaufmeldungen}
werden dann angezeigt, wenn das System eine l\"{a}ngere Bearbeitung
durchf\"{u}hrt. Diese Meldungen erfordern keine Bedienungsschritte des
Benutzers. Sie verschwinden nach vollendeter Bearbeitung. Sie erscheinen in der unteren Leiste des Fensters.\\

{\bf Benutzermeldungen} werden dann angezeigt, wenn der
Benutzer die ausgegebene Meldung quittieren mu{\ss} (z.B. ``Band
einlegen und eine beliebige Taste dr\"{u}cken''). Das Quittieren einer
solchen Meldung erfolgt durch Bet\"{a}tigung einer beliebigen Taste bzw.durch
Anklicken des Druckknopfes ``Schlie{\ss}en''.
